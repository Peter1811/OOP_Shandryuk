\documentclass[12pt]{article}

\usepackage{fullpage}
\usepackage{multicol,multirow}
\usepackage{tabularx}
\usepackage{ulem}
\usepackage[utf8]{inputenc}
\usepackage[russian]{babel}
\usepackage{minted}
\usepackage{color} %% это для отображения цвета в коде
\usepackage{listings} %% собственно, это и есть пакет listings
\lstset{ %
language=C++,                 % выбор языка для подсветки (здесь это С++)
basicstyle=\small\sffamily, % размер и начертание шрифта для подсветки кода
numbers=left,               % где поставить нумерацию строк (слева\справа)
%numberstyle=\tiny,           % размер шрифта для номеров строк
stepnumber=1,                   % размер шага между двумя номерами строк
numbersep=5pt,                % как далеко отстоят номера строк от подсвечиваемого кода
backgroundcolor=\color{white}, % цвет фона подсветки - используем \usepackage{color}
showspaces=false,            % показывать или нет пробелы специальными отступами
showstringspaces=false,      % показывать или нет пробелы в строках
showtabs=false,             % показывать или нет табуляцию в строках
frame=single,              % рисовать рамку вокруг кода
tabsize=2,                 % размер табуляции по умолчанию равен 2 пробелам
captionpos=t,              % позиция заголовка вверху [t] или внизу [b] 
breaklines=true,           % автоматически переносить строки (да\нет)
breakatwhitespace=false, % переносить строки только если есть пробел
escapeinside={\%*}{*)}   % если нужно добавить комментарии в коде
}


\begin{document}
\begin{titlepage}
\begin{center}
\textbf{МИНИСТЕРСТВО ОБРАЗОВАНИЯ И НАУКИ РОССИЙСКОЙ ФЕДЕРАЦИИ
\medskip
МОСКОВСКИЙ АВИАЦИОННЫЙ ИНСТИТУТ
(НАЦИОНАЛЬНЫЙ ИССЛЕДОВАТЕЛЬСКИЙ УНИВЕРСТИТЕТ)
\vfill\vfill
{\Huge ЛАБОРАТОРНАЯ РАБОТА №1} \\
по курсу объектно-ориентированное программирование
I семестр, 2021/22 уч. год}
\end{center}
\vfill

Студент \uline{\it {Шандрюк Пётр Николаевич, группа М8О-208Б-20}\hfill}

Преподаватель \uline{\it {Дорохов Евгений Павлович}\hfill}

\vfill
\end{titlepage}

\subsection*{Условие}

Задание: \
Вариант 25: Треугольник, квадрат, прямоугольник.\
Необходимо спроектировать и запрограммировать на языке C++ классы трех фигур, согласно варианту задания. Классы должны удовлетворять следующим правилам:
\begin{enumerate}
\item Должны быть названы также, как в вариантах задания и расположенны в раздельных файлах: отдельно заголовки (имя\_класса\_с\_маленькой\_буквы.h), отдельно описание методов (имя\_класса\_с\_маленькой\_буквы.cpp).
\item Иметь общий родительский класс Figure;
\item Содержать конструктор, принимающий координаты вершин фигуры из стандартного потока std::cin, расположенных через пробел. Пример: "0.0 0.0 1.0 0.0 1.0 1.0 0.0 1.0"
\item Содержать набор общих методов:
\begin{itemize}
    \item size\_t VertexesNumber() - метод, возвращающий количество вершин фигуры;
    \item double Area() - метод расчета площади фигуры;
    \item void Print(std::ostream& os) - метод печати типа фигуры и ее координат вершин в поток вывода os в формате: "Triangle: (0.0, 0.0) (1.0, 0.0) (1.0, 1.0)" с переводом строки в конце.
\end{itemize}
\end{enumerate}

\subsection*{Описание программы}

Исходный код лежит в 11 файлах:
\begin{enumerate}
\item main.cpp: основная программа, взаимодействие с пользователем посредством комманд из меню

\item figure.h:    описание абстрактного класса фигур

\item point.h:     описание класса точки
\item triangle.h:  описание класса треугольник, наследующегося от figures
\item square.h: описание класса квадрата, наследующегося от figures
\item rectangle.h:    описание класса прямоугольника, наследующегося от figures

\item point.cpp:     реализация класса точки
\item trinagle.cpp:  реализация класса треугольника, наследующегося от figures
\item square.cpp: реализация класса квадрата, наследующегося от figures
\item rectangle.cpp:    реализация класса прямоугольника, наследующегося от figures

\end{enumerate}

\subsection*{Дневник отладки}


\subsection*{Недочёты}


\subsection*{Выводы}

Я научился создавать и реализовывать классы в С++, а также познакомился с дружественными функциями, перегрузкой операторов и изучил основные понятия ООП.


\vfill

\subsection*{Исходный код}

{\Huge figure.h}
\inputminted{C++}{figure.h}
    \pagebreak
    
{\Huge point.h}
\inputminted{C++}{point.h}
    \pagebreak

{\Huge point.cpp}
\inputminted{C++}{point.cpp}
    \pagebreak

{\Huge triangle.h}
\inputminted{C++}{triangle.h}
\pagebreak

{\Huge triangle.cpp}
\inputminted{C++}{triangle.cpp}
\pagebreak

{\Huge rectangle.h}
\inputminted{C++}{rectangle.h}
\pagebreak

{\Huge rectangle.cpp}
\inputminted{C++}{rectangle.cpp}
\pagebreak

{\Huge square.h}
\inputminted{C++}{square.h}
\pagebreak

{\Huge square.cpp}
\inputminted{C++}{square.cpp}
\pagebreak
    
{\Huge main.cpp}
\inputminted{C++}{main.cpp}
    \pagebreak

\end{document}
